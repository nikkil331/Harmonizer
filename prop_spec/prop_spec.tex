\documentclass{sig-alternate}
 
\usepackage{mdwlist}
\usepackage{url}

\begin{document} 

% We setup the parameters to our title header before 'making' it. Note
% that your proposals should have actual titles, not the generic one
% we have here.
\title{Automatic Harmonization Viewed as a Machine Translation Problem}
\subtitle{Dept. of CIS - Senior Design 2014-2015\thanks{Advisor: Insup Lee (lee@cis.upenn.edu).}}
\numberofauthors{2}
\author{
\alignauthor Nicole Limtiaco \\ \email{limni@seas.upenn.edu} \\ Univ. of Pennsylvania \\ Philadelphia, PA
\alignauthor Rigel Swavely \\ \email{rigel@seas.upenn.edu} \\ Univ. of Pennsylvania \\ Philadelphia, PA}
\date{}
\maketitle

% Next we write out our abstract -- generally a two paragraph maximum,
% executive summary of the motivation and contributions of the work.
\begin{abstract}
  \textit{The purpose of this document is three-fold. First, it
    describes the requirements of the project proposal. Second, it
    presents this information in a style your proposal should
    mimic. Third, by viewing the *.tex `code behind', some basic
    \LaTeX{} technique can be learned. Do not feel compelled to
    completely follow our style -- but do use it as a guide.}

  \textit{Like this `report', your proposal is expected to have an
    abstract. An abstract is a two-paragraph maximum executive summary
    of your work. It should briefly outline your problem statement and
    its (expected) contributions.}
\end{abstract}

\section{Introduction}
\label{sec:intro}
Many would consider music composition to be an art form that is accomplished
primarily through human creativity. Writing music is a process that seems to require
complex thought which fundamentally cannot be boiled down to a list of straightforward
processes. Our project aims to answer the question, "Can computers imitate the complex
and creative process of composing music?" We narrow the question down to the more specific
and well-defined problem of generating harmony lines for a given melody line. Specifically,
we would like to create 4-part chorale settings given the first part, the melody.

We will define a \textit{melody} as a sequence of input \textit{notes}, where each note contains information
about pitch and timing. A \textit{harmony} will be a sequence of output notes which are produced with
the constraint that the sequence of notes support the input melody. We define the term \textit{voice}
to be some sequence of notes, either a melody or one of the harmonies, with a certain set of identifying 
characteristics. For example, the bass part in rock music is one type of voice and the soprano's part in an 
opera is another. The end result we wish to generate will be a group of 4 voices--the input melody and 
three automatically-composed harmonies--that, when played together, sound coherent and pleasant.

We propose to approach the problem from the view point of machine translation, where the input language
is the melody and the target language is the specific harmony part you wish to generate. The first analogy
that can be drawn between a pair of natural languages and a pair of melody and harmony voices is that both
have a sense of ``word translations''. Just as there may be several words in the target language that could
be sensible translations of a given word in the input langauge, there are also several notes in the harmony 
``language'' that can harmonize well with an input melody note. Importantly, however, it is not the case that
any note can harmonize well with the melody note. Some notes will sound dissonant when played together 
and still other notes may not even be in the harmony language, since harmony voices can have specified note
ranges. 

The second analogy that allows us to view this problem as a machine translation problem is that, like 
natural language, only certain strings of tokens (i.e. notes) are sensible in a given harmony voice. For example,
the statement ``colorless green ideas sleep furiously'' contains all english words but is unlikely to
be understood by an english speaker because the string of words in not sensible based on the rules of our
language. Similarly, a very inharmonious sequence of notes may not sound sensible in the context of its
harmony voice, if the notes are even recognized as music at all.

\section{Related Work}
\label{sec:related_work}
Automatic harmonization is a subset of the automatic musical composition problem,
which dates as far back as the field of artificial intelligence itself. Perhaps
the earliest work in automatic composition is Hiller and Isaacson's
\textit{Illiac Suite} \cite{hiller1959experimental}, which is widely accepted as being the first
musical piece composed by an electronic computer. Hiller and Isaacson
used a generate-and-test method that generated musical phrases psuedo-randomly and 
kept only those that adhered to a set of music-theory-inspired heuristics. 

Staying in line with the use of musical rules, Ebcio\v{g}lu \cite{Ebci̇oğlu1990145} provided a break-through 
in the specific field of automatic harmonization through his CHORAL system. CHORAL, 
bolstered by about 350 musical rules expressed in first order logic, performed the task of
writing four-part harmonies in the style J.S. Bach. With these logical predicates, Ebcio\v{g}lu 
reduced the problem of composing a harmony to a simple constraint satisfaction problem. Similar 
later works, notably by Tsang \& Aitkin \cite{tsang1991}, also crafted the harmonization problem as a 
problem in constraint satisfaction, but with a significantly smaller amount ($\sim$20) of musical rules.
The result of these constraint-based works were pieces of music that were non-offensive
to the ear in that there were no dissonant musical patterns allowed by the constraints; however,
creativity and overall similarity to human composition was still lacking.

More recent works began to put data-driven methods into use in order to create more human-like
compositions. A simple case-based model implemented by Sabater \textit{et al.} \cite{Sabater98usingrules} was built to generate
accompanying chords for a melody line. To a choose a harmony chord for a given context of previous
harmony chords and the currently sounding melody note, the system would check a case base to see if any
cases had the same harmony context and melody note and use the corresponding harmony chord if such a 
match were found. Musical heuristics were used to generate a chord if no match was found in the case base.
An automatic harmonizer utilizing neural networks was also built by Hild \textit{et al.} \cite{NIPS1991_576}  to produce a harmony
chord for a given melody quarter beat. Input features to the network included harmony context, current melody pitch,
and whether or not the beat was stressed.

As these examples show, the previous harmony context and the melody pitch are important signals in deciding
what the current harmony phrase should be. Many works have been conducted that model these signals using
n-gram Markov Models. A Markov model assigns probabilities to some event $C_{t}$ conditioned on a limited history
of previous events [$C_{t-1}...\\
C_{t-(n-1)}$], implictly making the assumption that the event $C_{t}$ depends only on a 
small amount of information about its context. For example, a system called MySong produced by Simon \textit{et al.} \cite{export:64277}
generates chord accompaniment given a vocalized melody by using a 2-gram Markov model for the harmony
context and a 2-gram Markov model for the melody context. A similar system implented by Scholz \textit{et al.} \cite{4959518},
which also generates chord accompanimants, experimented with 3-gram to 5-gram Markov models and incorporated
smoothing techniques commonly seen in NLP to account for cases in whcih the model has not seen a context
that is present in the test data. Most recently, Raczy\'{n}ski \textit{et al.} \cite{doi:10.1080/09298215.2013.822000} use discriminative Markov models that
model harmony context, melody context, and additionaly the harmony relationship to the tonality.

The recent Senior Design Project by Cerny \textit{et al.} \cite{UAMP} also used melody pitch and previous harmony context
as their main signals for determining the next harmony chord to generate. However, they used these
signals as inputs to an SVM classifier, as opposed to training a Markov model. 

\section{Project Proposal}
\label{sec:project_proposal}
Now is the time to introduce your proposed project in all of its
glory. Admittedly, this is not the easiest since you probably have not
done much actual research yet. Even so, setting and realizing
realistic research goals is an important skill. Begin by summarizing
what you are going to do and the expected benefit it will bring.

\subsection{Anticipated Approach}
\label{subsec:approach}
Having summarized \textit{what} you are going to do, its time to
describe \textit{how} you plan to do it. Our factorization example
does not work so well here (it is likely impossible to realize) -- so
let us suppose you are going to create a service that takes a
cell-phone picture of a building and returns via text-message, the
name of that building\footnote{Do not use this idea -- someone did it
  in a previous year.}.

In this case you might want to talk about establishing a server to
receive pictures via MMS. Once the picture is received, you will run
an edge extraction algorithm over it. Then, similarity between the
submitted picture and those stored (and tagged) in a MySQL database
will be computing using algorithm $XYZ$. Finally, the tag of the most
similar image will be returned to the user. Do not bore the reader
with trivial details, but give them an overview; a block-flow diagram
would prove helpful (and is required).

\subsection{Technical Challenges}
\label{subsec:tech_challenges}
In this subsection note where you anticipate having \underline{novel}
difficulty. Maybe you have never setup a MySQL database or even used
SQL before at all -- yes, that is a challenge -- but not one readers
care about. More novel would be the fact that many buildings on Penn's
campus look similar and your classifier may be inaccurate in such
instances. The purpose of this section is two-fold: 1) you will think
about which parts of your project would require the most time and
effort and 2) you will convince the reader that this is a project
worth undertaking.

\subsection{Evaluation Criteria}
\label{subsec:eval_criteria}
Suppose you have implemented your approach and it is functioning. Now
how are you going to convince readers your approach is better than
what exists? In the factorization example, you could just compare
run-times between algorithms run on the same input. The image
recognition example might use a percentage of accurate
classifications. Other fields may have established testing benchmarks.

No matter the case, you need to prove you have contributed to the
field. This will be easier for some than others. In particular, those
with `sensory' projects involving visual or sonic elements need to
think this point through -- objective measures are always better than
subjective ones.

\section{Research Timeline}
\label{sec:research_timeline}
We have completed some preliminary steps for our project and plan to complete
our work along this proposed timeline.

\begin{itemize*}
	\item {\sc already completed}: Preliminary reading. Experimented with musci21 library and corpus by implementing a deterministic harmony of thirds generator. Initial steps taken to train language model and translation models.\vspace{3pt}
	\item {\sc prior-to thanksgiving} : Finish training language model and translation model. Write beam search to find solution for one harmony voice.\vspace{3pt}
	\item {\sc prior-to christmas} : Tune beam size and n-gram size. Implement code to generate harmony voices in different orders and determine which order is best.\vspace{3pt}
	\item {\sc completion tasks} : Verify implementation is bug-free. Complete write-up. Get humans to evaluate output.\vspace{3pt}
	\item {\sc if there's time} : Investigate incorporating rhythm and timing decisions into the generated harmonies.
\end{itemize*}

\bibliographystyle{plain} % Please do not change the bib-style
\bibliography{prop_spec}  % Just the *.BIB filename

\end{document} 

